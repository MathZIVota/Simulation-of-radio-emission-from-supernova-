\RequirePackage[l2tabu,orthodox]{nag} % Раскомментировав, можно в логе получать рекомендации относительно правильного использования пакетов и предупреждения об устаревших и нерекомендуемых пакетах
% \documentclass[a4paper,14pt]{extarticle}
\usepackage[left=1.5 cm,right=1.6cm,top=1.5cm,bottom=2.5cm]{geometry}

%%% Mathematical packages %%%
\usepackage{amsthm,amsmath,amscd} % Математические дополнения от AMS
\usepackage{amsfonts,amssymb}     % Математические дополнения от AMS
\usepackage{mathtools}            % Добавляет окружение multlined
\usepackage{mathtext}
\usepackage{cancel}

\usepackage{textcomp}

\RequirePackage{ifxetex, ifluatex}
\ifxetex
  % https://tex.stackexchange.com/a/38631
  \renewcommand{\mathbf}{\ensuremath{\symbf}}
  \usepackage{unicode-math}
  \usepackage{polyglossia}                        % Поддержка многоязычности (fontspec подгружается автоматически)
  \setmainlanguage[babelshorthands=true]{russian} % Язык по-умолчанию русский с поддержкой приятных команд пакета babel
  \setotherlanguage{english}                      % Дополнительный язык = английский (в американской вариации по-умолчанию)
  % Семейство шрифтов Liberation (https://pagure.io/liberation-fonts)
  \setmonofont{LiberationMono}[Scale=0.87]        % моноширинный шрифт
  \newfontfamily\cyrillicfonttt{LiberationMono}[  % моноширинный шрифт для кириллицы
    Scale=0.87]
  \defaultfontfeatures{Ligatures=TeX}             % стандартные лигатуры TeX, замены нескольких дефисов на тире и т. п. Настройки моноширинного шрифта должны идти до этой строки, чтобы при врезках кода программ в коде не применялись лигатуры и замены дефисов
  \setmainfont{LiberationSerif}                   % Шрифт с засечками
  \newfontfamily\cyrillicfont{LiberationSerif}    % Шрифт с засечками для кириллицы
  \setsansfont{LiberationSans}                    % Шрифт без засечек
  \newfontfamily\cyrillicfontsf{LiberationSans}   % Шрифт без засечек для кириллицы

  % fake small capitals
  % https://tex.stackexchange.com/questions/55664/fake-small-caps-with-xetex-fontspec
  \makeatletter
  \newlength\fake@f
  \newlength\fake@c
  \def\textsc#1{%
    \begingroup%
    \xdef\fake@name{\csname\curr@fontshape/\f@size\endcsname}%
    \fontsize{\fontdimen8\fake@name}{\baselineskip}\selectfont%
    \MakeUppercase{#1}%
    \endgroup%
    }
  \makeatother
  % \renewcommand{\textsc}[1]{\fauxschelper#1 \relax\relax}
  % \def\fauxschelper#1 #2\relax{%
  %   \fauxschelphelp#1\relax\relax%
  %   \if\relax#2\relax\else\ \fauxschelper#2\relax\fi%
  %   }
  % \def\Hscale{.83}\def\Vscale{.72}\def\Cscale{1.00}
  % \def\fauxschelphelp#1#2\relax{%
  %   \ifnum`#1>``\ifnum`#1<`\{\scalebox{\Hscale}[\Vscale]{\uppercase{#1}}\else%
  %   \scalebox{\Cscale}[1]{#1}\fi\else\scalebox{\Cscale}[1]{#1}\fi%
  %   \ifx\relax#2\relax\else\fauxschelphelp#2\relax\fi}

\else
  \usepackage[T2A]{fontenc}           % кодировка
  \usepackage[utf8]{inputenc}         % Кодировка utf8
  \usepackage[english,russian]{babel} % Языки: русский, английский
\fi

\usepackage[colorlinks=true,unicode=true]{hyperref}

%%% Other packages %%%
\usepackage{xspace} % пробелы после предопределённых команд
\usepackage{color}
\usepackage{enumitem}
\usepackage{cmap}
\usepackage{array}
\usepackage{braket}
\usepackage{epsfig}
\usepackage{epstopdf}
\usepackage{graphicx}
\usepackage{float}
\usepackage{caption}
\captionsetup{compatibility=false}
\usepackage{subcaption}
\usepackage{indentfirst}
\usepackage{hyphenat}
\usepackage[normalem]{ulem}
\usepackage{wrapfig}
\usepackage{pdfpages}
\usepackage[normalem]{ulem}

\graphicspath{{img/}} % Пути к изображениям

\usepackage{multirow}
%%% Toc %%%
% \setcounter{tocdepth}{4}
% \setcounter{secnumdepth}{4}

%%% Title %%%
% \usepackage{titlesec}
% \titleformat{\section}
% {\normalfont\large\bfseries}{\thesection}{1em}{}

%%% Setup bibliography %%%

\usepackage{csquotes} % biblatex рекомендует его подключать. Пакет для оформления сложных блоков цитирования.
%%% Загрузка пакета с основными настройками %%%
\makeatletter
\usepackage[%
backend=biber,% движок
bibencoding=utf8,% кодировка bib файла
sorting=none,% настройка сортировки списка литературы
style=gost-numeric,% стиль цитирования и библиографии (по ГОСТ)
language=autobib,% получение языка из babel/polyglossia, default: autobib % если ставить autocite или auto, то цитаты в тексте с указанием страницы, получат указание страницы на языке оригинала
autolang=other,% многоязычная библиография
clearlang=true,% внутренний сброс поля language, если он совпадает с языком из babel/polyglossia
defernumbers=true,% нумерация проставляется после двух компиляций, зато позволяет выцеплять библиографию по ключевым словам и нумеровать не из большего списка
sortcites=true,% сортировать номера затекстовых ссылок при цитировании (если в квадратных скобках несколько ссылок, то отображаться будут отсортированно, а не абы как)
movenames=false, % опция разрешает или запрещает перемещение имён в область сведений об ответственности, если количество имён больше трёх.
% не менять местами заголовок и список авторов, если авторов больше четырех
minnames=3, % сокращение списка имён
maxnames=4, % сокращение списка имён
doi=true,% Показывать или нет ссылки на DOI
isbn=false,% Показывать или нет ISBN, ISSN, ISRN
url=false,
eprint=true,
backref=true
]{biblatex}[2016/09/17]
%]{biblatex}
%\ltx@iffilelater{biblatex-gost.def}{2017/05/03}%
{\toggletrue{bbx:gostbibliography}%
\renewcommand*{\revsdnamepunct}{\addcomma}}{}
\makeatother

\DefineBibliographyStrings{english}{docthesis = {dissertation}}
\DefineBibliographyStrings{russian}{docthesis = {диссертация}}

% Custom backref Text
%https://tex.stackexchange.com/questions/196015/custom-backref-text
\DefineBibliographyStrings{english}{
  backrefpage  = {Цит. на с.\adddot},
  backrefpages = {Цит. на с.\adddot},
}
\DefineBibliographyStrings{russian}{
  backrefpage  = {Цит. на с.\adddot},
  backrefpages = {Цит. на с.\adddot},
}
\ifxetex
\else
% Исправление случая неподдержки знака номера в pdflatex
    \DefineBibliographyStrings{russian}{number={\textnumero}}
\fi

% разделитель ; для ссылок
\DeclareMultiCiteCommand{\multicites}[\mkbibbrackets]{\cite}{\addsemicolon\space}

%%% Colors %%%
\usepackage[dvipsnames]{xcolor}

\definecolor{linkcolor}{rgb}{0.08, 0.38, 0.74}
\definecolor{citecolor}{rgb}{0.18, 0.55, 0.34}
\definecolor{urlcolor}{rgb}{0.03, 0.57, 0.82}

\hypersetup{
    linktocpage=true,           % ссылки с номера страницы в оглавлении, списке таблиц и списке рисунков
    colorlinks,                 % ссылки отображаются раскрашенным текстом, а не раскрашенным прямоугольником, вокруг текста
    linkcolor={linkcolor},      % цвет ссылок типа ref, eqref и подобных
    citecolor={citecolor},      % цвет ссылок-цитат
    urlcolor={urlcolor},        % цвет гиперссылок
}

%%% Users commands %%%

\def\stella{\code{STEL\-LA}\xspace}
\def\millimetron{\code{Миллиметрон}\xspace}
\def\mesa{\code{ME\-SA}\xspace}
\def\supremna{\code{SUP\-REM\-NA}\xspace}

\def\araa{Annual Review of Astronomy and Astrophysics}
\def\apj{The Astrophysical Journal}
\def\apjl{The Astrophysical Journal Letters}
\def\apjs{The Astrophysical Journal Supplement}
\def\apss{Astrophysics and Space Science}
\def\azh{Астрон. Журнал}
\def\pazh{Письма в Астрон. Журнал}
\def\pasp{Pub. Astron. Soc. Pacific}
\def\pasa{Pub. Astron. Soc. Australia}
\def\prl{Phys. Rev. Lett}
\def\pre{Phys. Rev. E}
\def\sovast{Soviet Astronomy}
\def\aa{Astronomy and Astrophysics}
\def\aapr{Astronomy and Astrophysics Reviews}
\def\aj{Astronomical Journal}
\def\mnras{MNRAS}
\def\nat{Nature}
\def\ssr{Space Science Reviews}
\def\prd{Phys. Rev. D}
\def\jqsrt{Journal of Quantitative Spectroscopy and Radiative Transfer}

\DeclareRobustCommand{\todo}{\textcolor{red}}

\newcommand{\code}[1]{\texttt{#1}}
% \newcommand{\code}[1]{\textsc{#1}}
\newcommand\vecx[1]{\ifstrequal{#1}{0}{\ensuremath{\mathbf{0}}}{\ensuremath{\boldsymbol{#1}}}}

\newcommand\vecxu{\vecx{u}}


\newcommand{\stkout}[1]{\ifmmode\text{\sout{\ensuremath{#1}}}\else\sout{#1}\fi}

\newcommand{\pb}[1]{\textbf{\color{magenta}PB: #1}}
%\newcommand{\pbc}[2]{\textbf{\stkout{#1} \pb{#2}}}
\newcommand{\pbc}[2]{\textbf{\stkout{#1}\color{magenta}#2}}
\newcommand{\pbd}[1]{\textbf{\stkout{#1}}}

\newcommand{\iz}[1]{\textbf{\color{orange}IZ: #1}}

\newcommand\nifsx{$^{56}$Ni\xspace}
\newcommand\cofsx{$^{56}$Co\xspace}
\newcommand\fefsx{$^{56}$Fe\xspace}
\newcommand{\rsun}{\ensuremath{R_\odot}\xspace}
\newcommand{\msun}{\ensuremath{M_\odot}}

% 
\def\rej{\ensuremath{R_{\rm ej}}}
\def\mej{\ensuremath{M_{\rm ej}}}
\def\renv{\ensuremath{R_{\rm env}}}
\def\menv{\ensuremath{M_{\rm env}}}

\newcommand\snia{SN\,Ia\xspace}
\newcommand\snib{SN\,Ib\xspace}
\newcommand\snic{SN\,Ic\xspace}
\newcommand\sniib{SN\,IIb\xspace}
\newcommand\sniip{SN\,IIP\xspace}