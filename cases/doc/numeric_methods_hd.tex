% arara: pdflatex
% arara: biber
% arara: pdflatex
% arara: pdflatex
% arara: clean: { files: [numeric_methods_hd.log,numeric_methods_hd.aux,numeric_methods_hd.blg,numeric_methods_hd.bbl,numeric_methods_hd.bcf,numeric_methods_hd.toc,numeric_methods_hd.run.xml,numeric_methods_hd.out ] }
% zsh> setopt no_nomatch

% Chandra Images by Category: Supernovas & Supernova Remnants
% https://chandra.harvard.edu/photo/category/snr.html

\documentclass[a4paper,12pt]{extarticle}
\usepackage[T2A]{fontenc}           % Кодировка (для pdflatex)
\usepackage[utf8]{inputenc}         % Кодировка utf8 (для pdflatex)
\usepackage[english,russian]{babel} % Языки: русский, английский
\RequirePackage[l2tabu,orthodox]{nag} % Раскомментировав, можно в логе получать рекомендации относительно правильного использования пакетов и предупреждения об устаревших и нерекомендуемых пакетах
% \documentclass[a4paper,14pt]{extarticle}
\usepackage[left=1.5 cm,right=1.6cm,top=1.5cm,bottom=2.5cm]{geometry}

%%% Mathematical packages %%%
\usepackage{amsthm,amsmath,amscd} % Математические дополнения от AMS
\usepackage{amsfonts,amssymb}     % Математические дополнения от AMS
\usepackage{mathtools}            % Добавляет окружение multlined
\usepackage{mathtext}
\usepackage{cancel}

\usepackage{textcomp}

\RequirePackage{ifxetex, ifluatex}
\ifxetex
  % https://tex.stackexchange.com/a/38631
  \renewcommand{\mathbf}{\ensuremath{\symbf}}
  \usepackage{unicode-math}
  \usepackage{polyglossia}                        % Поддержка многоязычности (fontspec подгружается автоматически)
  \setmainlanguage[babelshorthands=true]{russian} % Язык по-умолчанию русский с поддержкой приятных команд пакета babel
  \setotherlanguage{english}                      % Дополнительный язык = английский (в американской вариации по-умолчанию)
  % Семейство шрифтов Liberation (https://pagure.io/liberation-fonts)
  \setmonofont{LiberationMono}[Scale=0.87]        % моноширинный шрифт
  \newfontfamily\cyrillicfonttt{LiberationMono}[  % моноширинный шрифт для кириллицы
    Scale=0.87]
  \defaultfontfeatures{Ligatures=TeX}             % стандартные лигатуры TeX, замены нескольких дефисов на тире и т. п. Настройки моноширинного шрифта должны идти до этой строки, чтобы при врезках кода программ в коде не применялись лигатуры и замены дефисов
  \setmainfont{LiberationSerif}                   % Шрифт с засечками
  \newfontfamily\cyrillicfont{LiberationSerif}    % Шрифт с засечками для кириллицы
  \setsansfont{LiberationSans}                    % Шрифт без засечек
  \newfontfamily\cyrillicfontsf{LiberationSans}   % Шрифт без засечек для кириллицы

  % fake small capitals
  % https://tex.stackexchange.com/questions/55664/fake-small-caps-with-xetex-fontspec
  \makeatletter
  \newlength\fake@f
  \newlength\fake@c
  \def\textsc#1{%
    \begingroup%
    \xdef\fake@name{\csname\curr@fontshape/\f@size\endcsname}%
    \fontsize{\fontdimen8\fake@name}{\baselineskip}\selectfont%
    \MakeUppercase{#1}%
    \endgroup%
    }
  \makeatother
  % \renewcommand{\textsc}[1]{\fauxschelper#1 \relax\relax}
  % \def\fauxschelper#1 #2\relax{%
  %   \fauxschelphelp#1\relax\relax%
  %   \if\relax#2\relax\else\ \fauxschelper#2\relax\fi%
  %   }
  % \def\Hscale{.83}\def\Vscale{.72}\def\Cscale{1.00}
  % \def\fauxschelphelp#1#2\relax{%
  %   \ifnum`#1>``\ifnum`#1<`\{\scalebox{\Hscale}[\Vscale]{\uppercase{#1}}\else%
  %   \scalebox{\Cscale}[1]{#1}\fi\else\scalebox{\Cscale}[1]{#1}\fi%
  %   \ifx\relax#2\relax\else\fauxschelphelp#2\relax\fi}

\else
  \usepackage[T2A]{fontenc}           % кодировка
  \usepackage[utf8]{inputenc}         % Кодировка utf8
  \usepackage[english,russian]{babel} % Языки: русский, английский
\fi

\usepackage[colorlinks=true,unicode=true]{hyperref}

%%% Other packages %%%
\usepackage{xspace} % пробелы после предопределённых команд
\usepackage{color}
\usepackage{enumitem}
\usepackage{cmap}
\usepackage{array}
\usepackage{braket}
\usepackage{epsfig}
\usepackage{epstopdf}
\usepackage{graphicx}
\usepackage{float}
\usepackage{caption}
\captionsetup{compatibility=false}
\usepackage{subcaption}
\usepackage{indentfirst}
\usepackage{hyphenat}
\usepackage[normalem]{ulem}
\usepackage{wrapfig}
\usepackage{pdfpages}
\usepackage[normalem]{ulem}

\graphicspath{{img/}} % Пути к изображениям

\usepackage{multirow}
%%% Toc %%%
% \setcounter{tocdepth}{4}
% \setcounter{secnumdepth}{4}

%%% Title %%%
% \usepackage{titlesec}
% \titleformat{\section}
% {\normalfont\large\bfseries}{\thesection}{1em}{}

%%% Setup bibliography %%%

\usepackage{csquotes} % biblatex рекомендует его подключать. Пакет для оформления сложных блоков цитирования.
%%% Загрузка пакета с основными настройками %%%
\makeatletter
\usepackage[%
backend=biber,% движок
bibencoding=utf8,% кодировка bib файла
sorting=none,% настройка сортировки списка литературы
style=gost-numeric,% стиль цитирования и библиографии (по ГОСТ)
language=autobib,% получение языка из babel/polyglossia, default: autobib % если ставить autocite или auto, то цитаты в тексте с указанием страницы, получат указание страницы на языке оригинала
autolang=other,% многоязычная библиография
clearlang=true,% внутренний сброс поля language, если он совпадает с языком из babel/polyglossia
defernumbers=true,% нумерация проставляется после двух компиляций, зато позволяет выцеплять библиографию по ключевым словам и нумеровать не из большего списка
sortcites=true,% сортировать номера затекстовых ссылок при цитировании (если в квадратных скобках несколько ссылок, то отображаться будут отсортированно, а не абы как)
movenames=false, % опция разрешает или запрещает перемещение имён в область сведений об ответственности, если количество имён больше трёх.
% не менять местами заголовок и список авторов, если авторов больше четырех
minnames=3, % сокращение списка имён
maxnames=4, % сокращение списка имён
doi=true,% Показывать или нет ссылки на DOI
isbn=false,% Показывать или нет ISBN, ISSN, ISRN
url=false,
eprint=true,
backref=true
]{biblatex}[2016/09/17]
%]{biblatex}
%\ltx@iffilelater{biblatex-gost.def}{2017/05/03}%
{\toggletrue{bbx:gostbibliography}%
\renewcommand*{\revsdnamepunct}{\addcomma}}{}
\makeatother

\DefineBibliographyStrings{english}{docthesis = {dissertation}}
\DefineBibliographyStrings{russian}{docthesis = {диссертация}}

% Custom backref Text
%https://tex.stackexchange.com/questions/196015/custom-backref-text
\DefineBibliographyStrings{english}{
  backrefpage  = {Цит. на с.\adddot},
  backrefpages = {Цит. на с.\adddot},
}
\DefineBibliographyStrings{russian}{
  backrefpage  = {Цит. на с.\adddot},
  backrefpages = {Цит. на с.\adddot},
}
\ifxetex
\else
% Исправление случая неподдержки знака номера в pdflatex
    \DefineBibliographyStrings{russian}{number={\textnumero}}
\fi

% разделитель ; для ссылок
\DeclareMultiCiteCommand{\multicites}[\mkbibbrackets]{\cite}{\addsemicolon\space}

%%% Colors %%%
\usepackage[dvipsnames]{xcolor}

\definecolor{linkcolor}{rgb}{0.08, 0.38, 0.74}
\definecolor{citecolor}{rgb}{0.18, 0.55, 0.34}
\definecolor{urlcolor}{rgb}{0.03, 0.57, 0.82}

\hypersetup{
    linktocpage=true,           % ссылки с номера страницы в оглавлении, списке таблиц и списке рисунков
    colorlinks,                 % ссылки отображаются раскрашенным текстом, а не раскрашенным прямоугольником, вокруг текста
    linkcolor={linkcolor},      % цвет ссылок типа ref, eqref и подобных
    citecolor={citecolor},      % цвет ссылок-цитат
    urlcolor={urlcolor},        % цвет гиперссылок
}

%%% Users commands %%%

\def\stella{\code{STEL\-LA}\xspace}
\def\millimetron{\code{Миллиметрон}\xspace}
\def\mesa{\code{ME\-SA}\xspace}
\def\supremna{\code{SUP\-REM\-NA}\xspace}

\def\araa{Annual Review of Astronomy and Astrophysics}
\def\apj{The Astrophysical Journal}
\def\apjl{The Astrophysical Journal Letters}
\def\apjs{The Astrophysical Journal Supplement}
\def\apss{Astrophysics and Space Science}
\def\azh{Астрон. Журнал}
\def\pazh{Письма в Астрон. Журнал}
\def\pasp{Pub. Astron. Soc. Pacific}
\def\pasa{Pub. Astron. Soc. Australia}
\def\prl{Phys. Rev. Lett}
\def\pre{Phys. Rev. E}
\def\sovast{Soviet Astronomy}
\def\aa{Astronomy and Astrophysics}
\def\aapr{Astronomy and Astrophysics Reviews}
\def\aj{Astronomical Journal}
\def\mnras{MNRAS}
\def\nat{Nature}
\def\ssr{Space Science Reviews}
\def\prd{Phys. Rev. D}
\def\jqsrt{Journal of Quantitative Spectroscopy and Radiative Transfer}

\DeclareRobustCommand{\todo}{\textcolor{red}}

\newcommand{\code}[1]{\texttt{#1}}
% \newcommand{\code}[1]{\textsc{#1}}
\newcommand\vecx[1]{\ifstrequal{#1}{0}{\ensuremath{\mathbf{0}}}{\ensuremath{\boldsymbol{#1}}}}

\newcommand\vecxu{\vecx{u}}


\newcommand{\stkout}[1]{\ifmmode\text{\sout{\ensuremath{#1}}}\else\sout{#1}\fi}

\newcommand{\pb}[1]{\textbf{\color{magenta}PB: #1}}
%\newcommand{\pbc}[2]{\textbf{\stkout{#1} \pb{#2}}}
\newcommand{\pbc}[2]{\textbf{\stkout{#1}\color{magenta}#2}}
\newcommand{\pbd}[1]{\textbf{\stkout{#1}}}

\newcommand{\iz}[1]{\textbf{\color{orange}IZ: #1}}

\newcommand\nifsx{$^{56}$Ni\xspace}
\newcommand\cofsx{$^{56}$Co\xspace}
\newcommand\fefsx{$^{56}$Fe\xspace}
\newcommand{\rsun}{\ensuremath{R_\odot}\xspace}
\newcommand{\msun}{\ensuremath{M_\odot}}

% 
\def\rej{\ensuremath{R_{\rm ej}}}
\def\mej{\ensuremath{M_{\rm ej}}}
\def\renv{\ensuremath{R_{\rm env}}}
\def\menv{\ensuremath{M_{\rm env}}}

\newcommand\snia{SN\,Ia\xspace}
\newcommand\snib{SN\,Ib\xspace}
\newcommand\snic{SN\,Ic\xspace}
\newcommand\sniib{SN\,IIb\xspace}
\newcommand\sniip{SN\,IIP\xspace}


\makeatletter
\@ifundefined{c@basement}{
  \newcounter{basement}
  \setcounter{basement}{0} % 0 --- hide basement;
                            % 1 --- show basement
}{}
\makeatother

\hyphenation{
  smooth-ed
  par-tic-le
  hy-dro-dy-nam-ics
}


%%% Add bibliography
\addbibresource{refs_hd_gydro.bib}
\addbibresource{refs_hd.bib}

\begin{document}

%%% article title
\title{\large
 Заметки по численным методам, применяемым в задачах газодинамики при сильных разрывах
}

\author{Илья~Заворохин, П.В.~Бакланов}

\date{\today}

\maketitle

\tableofcontents


%%%%% BEGIN
%------------------------------------------
\section*{TODO}

\begin{enumerate}
    \item описать ключевые моменты по книге \cite{MolchanovGasdynamics2013}
    \item 1. Вид системы для применения чмов
    \item 2. Свойства численных схем, за которыми стоит следить
    \item 3. Сами численные схемы: РК4, Годунова, ...
    \item найти подходящие тесты
\end{enumerate}

%------------------------------------------
\newpage
\section{Введение}
    Цель работы - применение классических разностных методов (в частности метод Рунге Кутты 4-го порядка) к поиску приближённых решений гиперболических систем при наличии разрывов в начальных параметрах, знакомство с классическим методом Годунова и производными от него методами. Повышение точности численного решения применением схем годуновского типа к исходным методам (в частности, пересчет значений потоков по Годунову). Написание программной реализации этих методов.

\section{Исходная система}
\subsection{В общем виде}
    Полная классическая система газодинамики записывается в недивергентном виде:
    \begin{align} 
        &\frac{\partial \rho}{\partial t} + div(\rho \vecxu) = 0 \label{eq:continuity}\\
        %
        &\frac{\partial \vecxu}{\partial t} + \left(\vecxu \nabla \right)\vecxu = -\frac{1}{\rho}\nabla P  \label{eq:euler}	\\
        %
        &\frac{\partial }{\partial t}\rho E + div(\rho \vecxu (w+\frac{\vecxu^2}{2})) = 0  \label{eq:energy} \\
            %
        &p = p(\rho,e)
    \end{align}
    Используемые обозначения: $\rho$ - плотность, $u$ -скорость, $p$ -давление, $E = e + \frac{\vecxu^2}{2}$ - полная удельная (на единицу массы) энергия, $e$ - удельная (на единицу массы)внутренняя энергия, $w$ - удельная (на единицу массы) энтальпия.
    В таком виде она представлена в большинстве учебников по теории газовой динамики, например в \cite{godunov1976}. 
    Однако для применения численных методов ее преобразуют к другому виду. Для этого уравнение непрерывности(\ref{eq:continuity}) cначала умножается на U, а затем на E. Полученные уравнения складываются соответсвенно с уравением Эйлера (\ref{eq:euler}) и уравнением для энергии (\ref{eq:energy}). Учёт формул производной произведения позволяет привести эти уравнения к консервативному виду. Подобное преобразование указано например в \cite{MolchanovGasdynamics2013}, поэтому здесь приведем лишь полученную в результате дивергентную форму системы уравнений газовой динамики: 
    \begin{align}
        \frac{\partial \rho}{\partial t} + \frac{\partial (\rho u)}{\partial x} &= 0 \\
        \frac{\partial u}{\partial t} + u \frac{\partial u}{\partial x} &= - \frac{1}{\rho} \frac{\partial P}{\partial x} \\
        \frac{\partial \rho E}{\partial t} + \frac{\partial (\rho u(E + P))}{\partial x} &= 0 \\
        p &= p(\rho, e)
    \end{align}
    
    Во многих естественных задачах можно считать, что пространственная размерность задачи равна 1. Поэтому далее приводится вид системы для такого случая. 
\subsection{Одномерный вид}
    Если пространственная одномерность связана с одной из декартовых осей, то системы принимает вид:
    \begin{align}
        \frac{\partial \rho}{\partial t} + \frac{\partial (\rho u)}{\partial x} &= 0 \\
        \frac{\partial (\rho u)}{\partial t} + \frac{\partial (\rho u^2 + p)}{\partial x} &= 0 \\
        \frac{ \partial (\rho E)}{\partial t} + \frac{\partial \rho (u(E+p))}{\partial x} &= 0
    \end{align}

\subsection{Cферически симметричный вид}
    В некоторых задачах (в особенности при моделировании взрывов, в том числе вспышек сверхновых) возможно сведение задачи к сферически симметричному случаю. В этом случае пространственной координатой будет радиальное расстояние от центра сферы. В таком случае система примет вид:
    \begin{align}
        \frac{\partial \rho}{\partial t} + \frac{1}{r^2}\frac{\partial (r^2 (\rho u))}{\partial r} &= 0 \\
        \frac{\partial \rho u}{\partial t} + \frac{1}{r^2}\frac{\partial (r^2(\rho u^2+p))}{\partial r} &= 2\frac{p}{r} \\
        \frac{\partial (\rho E)}{\partial t} + \frac{1}{r^2}\frac{\partial(r^2 \rho u( E + p))}{\partial r} &= 0
    \end{align}

\section{Основные свойства численных схем}
\subsection{Метод контрольного объёма}
\subsection{Cходимость, порядок точности}
\subsection{Невязка, аппроксимация}
\subsection{Устойчивость}
\subsection{Требования для численных методов в случае решения задач газодинамики}
\subsection{Типы граничных условий}

\section{Численные методы}
\subsection{Метод Рунге-Кутты 4-го порядка}
\subsection{Базовый метод Годунова}
\subsection{Методы годуновского типа: Рое, WENO,....}

\section{Тесты}
\subsection{Cода}
\subsection{Тест с наличием скоростей}
\subsection{Тест с сильным разрывом (1000 и более раз)}

\section{Заключение}

%-------------------------------------------
\clearpage
\printbibliography
\end{document}