 
% Chandra Images by Category: Supernovas & Supernova Remnants
% https://chandra.harvard.edu/photo/category/snr.html

\documentclass[a4paper,12pt]{extarticle}
%\documentclass{article}
\usepackage{graphicx}
\RequirePackage[l2tabu,orthodox]{nag} % Раскомментировав, можно в логе получать рекомендации относительно правильного использования пакетов и предупреждения об устаревших и нерекомендуемых пакетах
% \documentclass[a4paper,14pt]{extarticle}
\usepackage[left=1.5 cm,right=1.6cm,top=1.5cm,bottom=2.5cm]{geometry}

%%% Mathematical packages %%%
\usepackage{amsthm,amsmath,amscd} % Математические дополнения от AMS
\usepackage{amsfonts,amssymb}     % Математические дополнения от AMS
\usepackage{mathtools}            % Добавляет окружение multlined
\usepackage{mathtext}
\usepackage{cancel}

\usepackage{textcomp}

\RequirePackage{ifxetex, ifluatex}
\ifxetex
  % https://tex.stackexchange.com/a/38631
  \renewcommand{\mathbf}{\ensuremath{\symbf}}
  \usepackage{unicode-math}
  \usepackage{polyglossia}                        % Поддержка многоязычности (fontspec подгружается автоматически)
  \setmainlanguage[babelshorthands=true]{russian} % Язык по-умолчанию русский с поддержкой приятных команд пакета babel
  \setotherlanguage{english}                      % Дополнительный язык = английский (в американской вариации по-умолчанию)
  % Семейство шрифтов Liberation (https://pagure.io/liberation-fonts)
  \setmonofont{LiberationMono}[Scale=0.87]        % моноширинный шрифт
  \newfontfamily\cyrillicfonttt{LiberationMono}[  % моноширинный шрифт для кириллицы
    Scale=0.87]
  \defaultfontfeatures{Ligatures=TeX}             % стандартные лигатуры TeX, замены нескольких дефисов на тире и т. п. Настройки моноширинного шрифта должны идти до этой строки, чтобы при врезках кода программ в коде не применялись лигатуры и замены дефисов
  \setmainfont{LiberationSerif}                   % Шрифт с засечками
  \newfontfamily\cyrillicfont{LiberationSerif}    % Шрифт с засечками для кириллицы
  \setsansfont{LiberationSans}                    % Шрифт без засечек
  \newfontfamily\cyrillicfontsf{LiberationSans}   % Шрифт без засечек для кириллицы

  % fake small capitals
  % https://tex.stackexchange.com/questions/55664/fake-small-caps-with-xetex-fontspec
  \makeatletter
  \newlength\fake@f
  \newlength\fake@c
  \def\textsc#1{%
    \begingroup%
    \xdef\fake@name{\csname\curr@fontshape/\f@size\endcsname}%
    \fontsize{\fontdimen8\fake@name}{\baselineskip}\selectfont%
    \MakeUppercase{#1}%
    \endgroup%
    }
  \makeatother
  % \renewcommand{\textsc}[1]{\fauxschelper#1 \relax\relax}
  % \def\fauxschelper#1 #2\relax{%
  %   \fauxschelphelp#1\relax\relax%
  %   \if\relax#2\relax\else\ \fauxschelper#2\relax\fi%
  %   }
  % \def\Hscale{.83}\def\Vscale{.72}\def\Cscale{1.00}
  % \def\fauxschelphelp#1#2\relax{%
  %   \ifnum`#1>``\ifnum`#1<`\{\scalebox{\Hscale}[\Vscale]{\uppercase{#1}}\else%
  %   \scalebox{\Cscale}[1]{#1}\fi\else\scalebox{\Cscale}[1]{#1}\fi%
  %   \ifx\relax#2\relax\else\fauxschelphelp#2\relax\fi}

\else
  \usepackage[T2A]{fontenc}           % кодировка
  \usepackage[utf8]{inputenc}         % Кодировка utf8
  \usepackage[english,russian]{babel} % Языки: русский, английский
\fi

\usepackage[colorlinks=true,unicode=true]{hyperref}

%%% Other packages %%%
\usepackage{xspace} % пробелы после предопределённых команд
\usepackage{color}
\usepackage{enumitem}
\usepackage{cmap}
\usepackage{array}
\usepackage{braket}
\usepackage{epsfig}
\usepackage{epstopdf}
\usepackage{graphicx}
\usepackage{float}
\usepackage{caption}
\captionsetup{compatibility=false}
\usepackage{subcaption}
\usepackage{indentfirst}
\usepackage{hyphenat}
\usepackage[normalem]{ulem}
\usepackage{wrapfig}
\usepackage{pdfpages}
\usepackage[normalem]{ulem}

\graphicspath{{img/}} % Пути к изображениям

\usepackage{multirow}
%%% Toc %%%
% \setcounter{tocdepth}{4}
% \setcounter{secnumdepth}{4}

%%% Title %%%
% \usepackage{titlesec}
% \titleformat{\section}
% {\normalfont\large\bfseries}{\thesection}{1em}{}

%%% Setup bibliography %%%

\usepackage{csquotes} % biblatex рекомендует его подключать. Пакет для оформления сложных блоков цитирования.
%%% Загрузка пакета с основными настройками %%%
\makeatletter
\usepackage[%
backend=biber,% движок
bibencoding=utf8,% кодировка bib файла
sorting=none,% настройка сортировки списка литературы
style=gost-numeric,% стиль цитирования и библиографии (по ГОСТ)
language=autobib,% получение языка из babel/polyglossia, default: autobib % если ставить autocite или auto, то цитаты в тексте с указанием страницы, получат указание страницы на языке оригинала
autolang=other,% многоязычная библиография
clearlang=true,% внутренний сброс поля language, если он совпадает с языком из babel/polyglossia
defernumbers=true,% нумерация проставляется после двух компиляций, зато позволяет выцеплять библиографию по ключевым словам и нумеровать не из большего списка
sortcites=true,% сортировать номера затекстовых ссылок при цитировании (если в квадратных скобках несколько ссылок, то отображаться будут отсортированно, а не абы как)
movenames=false, % опция разрешает или запрещает перемещение имён в область сведений об ответственности, если количество имён больше трёх.
% не менять местами заголовок и список авторов, если авторов больше четырех
minnames=3, % сокращение списка имён
maxnames=4, % сокращение списка имён
doi=true,% Показывать или нет ссылки на DOI
isbn=false,% Показывать или нет ISBN, ISSN, ISRN
url=false,
eprint=true,
backref=true
]{biblatex}[2016/09/17]
%]{biblatex}
%\ltx@iffilelater{biblatex-gost.def}{2017/05/03}%
{\toggletrue{bbx:gostbibliography}%
\renewcommand*{\revsdnamepunct}{\addcomma}}{}
\makeatother

\DefineBibliographyStrings{english}{docthesis = {dissertation}}
\DefineBibliographyStrings{russian}{docthesis = {диссертация}}

% Custom backref Text
%https://tex.stackexchange.com/questions/196015/custom-backref-text
\DefineBibliographyStrings{english}{
  backrefpage  = {Цит. на с.\adddot},
  backrefpages = {Цит. на с.\adddot},
}
\DefineBibliographyStrings{russian}{
  backrefpage  = {Цит. на с.\adddot},
  backrefpages = {Цит. на с.\adddot},
}
\ifxetex
\else
% Исправление случая неподдержки знака номера в pdflatex
    \DefineBibliographyStrings{russian}{number={\textnumero}}
\fi

% разделитель ; для ссылок
\DeclareMultiCiteCommand{\multicites}[\mkbibbrackets]{\cite}{\addsemicolon\space}

%%% Colors %%%
\usepackage[dvipsnames]{xcolor}

\definecolor{linkcolor}{rgb}{0.08, 0.38, 0.74}
\definecolor{citecolor}{rgb}{0.18, 0.55, 0.34}
\definecolor{urlcolor}{rgb}{0.03, 0.57, 0.82}

\hypersetup{
    linktocpage=true,           % ссылки с номера страницы в оглавлении, списке таблиц и списке рисунков
    colorlinks,                 % ссылки отображаются раскрашенным текстом, а не раскрашенным прямоугольником, вокруг текста
    linkcolor={linkcolor},      % цвет ссылок типа ref, eqref и подобных
    citecolor={citecolor},      % цвет ссылок-цитат
    urlcolor={urlcolor},        % цвет гиперссылок
}

%%% Users commands %%%

\def\stella{\code{STEL\-LA}\xspace}
\def\millimetron{\code{Миллиметрон}\xspace}
\def\mesa{\code{ME\-SA}\xspace}
\def\supremna{\code{SUP\-REM\-NA}\xspace}

\def\araa{Annual Review of Astronomy and Astrophysics}
\def\apj{The Astrophysical Journal}
\def\apjl{The Astrophysical Journal Letters}
\def\apjs{The Astrophysical Journal Supplement}
\def\apss{Astrophysics and Space Science}
\def\azh{Астрон. Журнал}
\def\pazh{Письма в Астрон. Журнал}
\def\pasp{Pub. Astron. Soc. Pacific}
\def\pasa{Pub. Astron. Soc. Australia}
\def\prl{Phys. Rev. Lett}
\def\pre{Phys. Rev. E}
\def\sovast{Soviet Astronomy}
\def\aa{Astronomy and Astrophysics}
\def\aapr{Astronomy and Astrophysics Reviews}
\def\aj{Astronomical Journal}
\def\mnras{MNRAS}
\def\nat{Nature}
\def\ssr{Space Science Reviews}
\def\prd{Phys. Rev. D}
\def\jqsrt{Journal of Quantitative Spectroscopy and Radiative Transfer}

\DeclareRobustCommand{\todo}{\textcolor{red}}

\newcommand{\code}[1]{\texttt{#1}}
% \newcommand{\code}[1]{\textsc{#1}}
\newcommand\vecx[1]{\ifstrequal{#1}{0}{\ensuremath{\mathbf{0}}}{\ensuremath{\boldsymbol{#1}}}}

\newcommand\vecxu{\vecx{u}}


\newcommand{\stkout}[1]{\ifmmode\text{\sout{\ensuremath{#1}}}\else\sout{#1}\fi}

\newcommand{\pb}[1]{\textbf{\color{magenta}PB: #1}}
%\newcommand{\pbc}[2]{\textbf{\stkout{#1} \pb{#2}}}
\newcommand{\pbc}[2]{\textbf{\stkout{#1}\color{magenta}#2}}
\newcommand{\pbd}[1]{\textbf{\stkout{#1}}}

\newcommand{\iz}[1]{\textbf{\color{orange}IZ: #1}}

\newcommand\nifsx{$^{56}$Ni\xspace}
\newcommand\cofsx{$^{56}$Co\xspace}
\newcommand\fefsx{$^{56}$Fe\xspace}
\newcommand{\rsun}{\ensuremath{R_\odot}\xspace}
\newcommand{\msun}{\ensuremath{M_\odot}}

% 
\def\rej{\ensuremath{R_{\rm ej}}}
\def\mej{\ensuremath{M_{\rm ej}}}
\def\renv{\ensuremath{R_{\rm env}}}
\def\menv{\ensuremath{M_{\rm env}}}

\newcommand\snia{SN\,Ia\xspace}
\newcommand\snib{SN\,Ib\xspace}
\newcommand\snic{SN\,Ic\xspace}
\newcommand\sniib{SN\,IIb\xspace}
\newcommand\sniip{SN\,IIP\xspace}



%%% Add bibliography
\addbibresource{refs_hd.bib}

\begin{document}
\begin{titlepage} 
    \begin{figure}[!htb]
	\centering
	\includegraphics[width=0.9\textwidth]{Sketch_MSU.png}
	\label{fig:Sketch_MSU}
    \end{figure}
    \begin{center}
    \textbf{\Large  МОСКОВСКИЙ ГОСУДАРСТВЕННЫЙ УНИВЕРСИТЕТ \\
ИМЕНИ М. В. ЛОМОНОСОВА} 
    \end{center}
    \begin{center}
    \end{center}
    \begin{center}
     \text{\Large Факультет космических исследований}
    \end{center}
    \begin{center}
    \end{center}
    \begin{center}
    \end{center}
    \begin{center}
    \text{\Large Курсовая работа\newline}
    
    \text{\Large 
    Моделирование радиоизлучения от сверхновых и их остатков}
    \end{center}
    \vspace{6cm}
    \begin{flushright}
    Выполнил студент 3 курса специалитета Заворохин Илья Владимирович\\
    Научный руководитель: к. ф-м. н. Бакланов Петр Валерьевич
    \end{flushright}
    \vspace{3cm}
    \begin{center}
        Москва, 2024
    \end{center}
\end{titlepage} 


\date{\today}
\tableofcontents
\newpage
\section{Введение}
Исследование сверхновых оказывает огромное влияние на развитие представлений учёных о фундаментальных явлениях происходящих в звёздах, о происхождении тяжёлых элементов, а также для исследования динамики эволюции всей Вселенной. 
Наблюдения сверхновых с Земли начались ещё за несколько тысячелетий до нашей эры.
\pb{todo: тут вставить мысль, что исследования СН начались давно. 
	Активно велись со времён китайских летописей,
дневников Тихо Браге. Важно, что они были все в оптике.}
Cущественные продвижения начались в середине 20 века, с развитием радиоастрономических наблюдений.\pb{todo: узнать дату первых радионаблюдений?}

В данной работе ставится целью изучение процессов, происходящих во время взрыва сверхновых.
Знакомство с явлением ударных волн (УВ) началось с написания программной реализации аналитического решения для известной задачи Римана о распаде произвольного разрыва в ударной трубе. Эта часть работы описана в разделе \ref{sec: Shockwave}.
%
Механизмы возникновения радиоизлучения в космосе перечислены в разделе \ref{sec: Radio emission}.
В разделе \ref{sec: Synchrotron} дается подробное описание синхротронного излучения, как основного реализуемого в остатке сверхновой, находящимся на поздней стадии разлёта (пройденное время с момента взрыва порядка 100-1000 дней) и значительно взаимодействующем с межзвёздным веществом. 

%---------------------------------------------------

\section{Сверхновые}
Сверхновая является заключительной стадией жизни некоторых звёзд.{\cite{Shklov1984}} В зависимости от внутреннего состава звезды вспышка может осуществляться разными механизмами и давать при этом разную картину на получаемых данных (кривых блеска). Выделяют 2 основных типа сверхновых: I и II. К I типу относят сверхновые, спектры которых не содержат линий водорода, ко II типу, наоборот, содержащие такие линии. Сверхновые I типа делят на 3 подтипа: Ia(есть кремний), Ib(есть гелий), Ic(нет гелия), а у II типа выделяют: 1) по спектру: IIb, IIn; 2)по кривой блеска(наличие плато): IIP и IIL. 
Сверхновые Ia существенно отличаются \pb{todo: чем? какие наблюдательные отличия?} \iz{здесь я скорее вел к тому, что у них происходит теормоядерный взрыв, и после него ничего не остается, а у других по-другому(то, что идет после двоеточия), добавил предложение про одиноковость их кривых блеска} от всех остальных: 
считается что к нему приводит термоядерный взрыв, а ко всем остальным коллапс ядра массивной звезды. Это приводит к тому, что наблюдаемый кривые блеска для всех сверхновых типа Ia одинаковы. 

Сброшенная при вспышке сверхновой оболочка расширяется со сверхзвуковой скоростью в межзвёздную среду и образует ударную волну. 
Различают несколько стадий взаимодействия оболочки с окружающей средой({\cite{Spitzer1981}}): свободный разлёт, адиабатическое расширение, стадия снегоочистителя. Далее идёт более подробное описание каждой из них.

\subsection{Стадия 1. Свободный разлёт}
На этой стадии оболочка движется по инерции так, как если бы внешней среды не было вообще. $R(t)\propto t $ 
Излучение оболочки не играет роли в ее динамике. Стадия заканчивается при сгребании массы окружающего вещества, равной массе расширяющейся оболочки $M_0 = 4\pi/3\rho_0R^3$. 
Сделаем оценку по порядку величины для характерных значений масс и скоростей разлетающейся оболочки сверхновой.
Для $\rho_0=2\times10^{-24}$\,г/см$^3$, $v=5\cdot10^8$\,см/с и $M_0=1M_{\odot}$ этот момент наступает примерно через 200 лет после начала расширения при $R\approx 2$\,пк.

\subsection{Стадия 2. Адиабатическое расширение} 
Радиационные процессы по-прежнему динамически неважны (отсюда название - адиабатическая стадия), так как температура газа за фронтом ударной волны очень высокая. Кинетическая энергия оболочки расходуется на нагрев газа за фронтом сильной ударной волны и на ускорение сгребённого межзвёздного газа. Когда масса сгребённого газа много больше $M_0$, движение оболочки довольно точно описывается автомодельным решением Л.И. Седова для сильного взрыва в среде. Можно получить зависимость поведения радиуса оболочки от времени из простых физических соображений.

Пусть тепловая энергия газа, находящаяся в равновесии с кинетической, составляет долю $K_1$ от полной энергии $E$, а давление непосредственно за фронтом УВ $p_2$ в $K_2$ раз больше среднего давления внутри оболочки. Для идеального газа с показателем адиабаты $\gamma=5/3$, среднее давление равно $p=(\gamma-1)\epsilon=2/3\epsilon,\  \epsilon$-средняя плотность тепловой энергии, что даёт 
$$p_2=K_2\cdot \frac{2}{3}\cdot \frac{3K_1E}{4\pi R_s^3}=\frac{KE}{2\pi R_s^3},\ \text{где } K=K_1K_2$$

Но в случае сильных ударных волн справедливо соотношение
$$p_2=\frac{2\rho_1u_1^2}{\gamma+1}$$
между давлением сразу за фронтом $p_2$, плотностью $\rho_1$ и скоростью втекания невозмущенного гаща в УВ $u_1$. Комбинирую эти уравнения и учитывая, что $u_1=dR_s/dt$, получаем $$u_1^2=\left(\frac{dR_s}{dt}\right)^2=\frac{2KE}{3\pi \rho_1 R_s^3}$$
Точный динамический расчет дает для $\gamma=5/3$ $K_1=0.72,K_2=2.13$, следовательно, $K=1.53$
Интегрируя последнее уравнение, получаем 
$$R_s(t)=\left(\frac{2.02E}{\rho_1}\right)^{\frac{1}{5}}\cdot t^{\frac{2}{5}}=\frac{0.26t^{\frac{2}{5}}}{n_1^{\frac{1}{5}}} \text{пк}$$,
где $n_1$ - концентрация атомов в невозмущенной межзвездной среде, время t выражено в годах, а численные коэффициенты получены при $E=4\cdot 10^{50}$эрг и $\rho_1=1.26m_Hn_1$ 
Поскольку температура за фронтом сильной ударной волны для идеального газа с $\gamma=5/3$
$$T_s = \frac{3\mu}{16k_B}u_1^2=1.8\cdot 10^5 \left(\frac{R}{t} \right)^2\text{кэВ},$$
где $k_B$-постоянная Больцмана, $\mu$ -молекулярный вес,
подставляя в это выражение полученные выше соотношения, получаем, что температура падает со временем, как 
$T \propto R_s^{-3} \propto t^{-\frac{6}{5}}$, начиная с некоторого момента времени (радиуса оболочки) становятся важными процессы радиоактивного охлаждения УВ и адиабатическое приближение нарушается. 
В конце стадии свободного разлета возникает обратная ударная волна, распространяющаяся внутрь оболочки(в системе координат, связанной с фронтом УВ), но движущаяся наружу в лабораторной системе (т.е газ втекает в обратную ударную волну изнутри оболчки). 

\subsection{Стадия 3. Стадия снегоочистителя}
Наступает после катастрофического охлаждения газа оболочки, когда температура падает ниже $\approx 6\times 10^5$ K и плазма начинает интенсивно высвечивать запасенную тепловую энергию. УВ при этом становится изотермической ($\gamma=1$). Оболочка становится тонкой и холодной, поскольку скорость газа, прошедшего через ударную волну, меньше скорости движения фронта по среде и газ, поджимаемый давлением оболочки изнутри, долго остается вблизи фронта УВ. Переход к этому режиму происходит при радиусе оболочки $$R_c=24\left( \frac{E\cdot10^{-51}\hbox{эрг/с}}{n_0}\right)^{\frac{1}{3}}\hbox{пк}$$
Движение УВ поддерживается за счет запасенного в оболочке импульса ($M(dR_s/dt)=const$, $M=4\pi/3\rho_1R_s^3$. В этом режиме расширение оболочки замедляется, т.к из сохранения импульса следует $dR_s/dt\propto 1/R_s^3$ (а не $R_s^{-3/2}$ как в случае адиабатического разлета). 

%------------------------------------------------

\section{Ударные волны}\label{sec: Shockwave}
Сильные скачки уплотнения идущие от центра звезды являются основной причиной взрывов сверхновых. На стадии адиабатического расширения остатка при нагребении им межзвездного газа массой, существенной превышающей массу самого остатка, его движение очень точно описывается автомодельным решением задачи Седова для сильного взрыва в среде(\cite{Sedov1977}).


\subsection{Основные уравнениия}
Выпишем систему уравнений газодинамики, определяющих изменение свойств газа: плотность, скорость и давление \cite{zr1968}.
Ее составляют уравнения для законов созранения массы, импульса и энергии соответсвенно.

\begin{align} 
	\frac{\partial \rho}{\partial t} + div(\rho \vecxu) = 0 \label{eq:continuity}\\
	%
	\frac{\partial  \vecxu}{\partial t} + \left(\vecxu \nabla \right)\vecxu = -\frac{1}{\rho}\nabla P  \label{eq:euler}	\\
	%
	\frac{\partial }{\partial t}\rho(E+\frac{\vecxu^2}{2}) + div(\rho \vecxu (w+\frac{\vecxu^2}{2})) = 0  \label{eq:energy} 
        %
\end{align}
Помимо этого выполнено уравнение состояния:
\begin{equation}
p = p(\rho,e)
\end{equation}
а также уравнение выражающее закон сохранения энтропии:
\begin{equation} \label{eq:entropy} 
\frac{d s}{d t} = 0
    \quad\text{или}\quad  
\frac{\partial s}{\partial t} + (\vecxu \cdot \nabla)s = 0 
\end{equation}

\subsection{Задача Римана о распаде произвольного разрыва}

Расматривается однмерная задача с двумя областями с различными идеальными газами, разделённых жесткой перегородкой. Начальные условия у газов различны, слева от перегородки $p_L, u_L, \rho_L$; справа - $p_R, u_R, \rho_R$.В момент $t=0 $ перегородка исчезает (\cite{BulatVolkov2015,zr1968}).

В этом случае исходная система дифференциальных уравнений (\ref{eq:continuity}-\ref{eq:energy}) преобразуется в систему нелинейных алгебраических уравнений. В нее не входят параметры с размерностями длины и времени, поэтому решение задачи является автомодельным, т.е такое, что переменные x и t входят в него лишь в комбинации \(x/t\).
В зависимости от начальных условий образуется та или иная
конфигурация устойчивых разрывов и непрерывных газодинамических течений (см. рис. 1). Возможные
решения содержат веер волн разрежения(ВР), контактный разрыв(КР) и ударную волну (УВ), разделяющие область
течения на четыре подобласти с постоянными значениями параметров. В конфигурации A возникает
ударная волна, контактный разрыв и веер волн разрежения, в конфигурации B – две ударные волны и
контактный разрыв, в конфигурации C – две волны разрежения и контактный разрыв. Условно волны
называют левой и правой волной (в неподвижной системе координат они могут двигаться в одну сторону). В случае ударной волны речь идет о движущемся фронте разрыва, по обе стороны которого параметры газа полагаются постоянными (своими для каждой из сторон) и связанными определенными соотношениями. В случае волны разрежения имеется область переменного течения, в которой параметры газа
остаются постоянными вдоль прямолинейных лучей, играющих роль характеристик системы уравнений,
а значения параметров зависят от наклона в веере характеристик, описывающем волну разрежения. Волна разрежения граничит с областями постоянного течения, подобным тем, которые имеют место для
ударной волны. 

В данной работе в качетсве теста используется задача Сода, начальные условия которой приводят к конфигурации А: ударная волна(справа), контактный разрыв, волна разрженеия(слева).
\begin{figure}[!htb]
	\centering
	\includegraphics[width=0.7\textwidth]{godunov1976_fig13-1.png}
	\caption{
		Воозможные конфигурации системы.
	}
	\label{fig:configurations}
\end{figure}

Для обоих газов выполнено уравнение состояние идеального газа: $ pV = \nu RT$, 

а также уравнение для энтропии идеального газа: $S = C_v\log(p/\rho^\gamma)+const$, 

показатель адиабаты обоих газов: $\gamma = 1.4$.

Используются следующие обозначения для состояний газа:
\begin{itemize}
    \item $p_L, u_L, \rho_L$ - начальные условия газа слева (газ, по которому еще не прошла волна разрежения),
    \item $p_1, u_1, \rho_1$ - условия в газе, охваченном волной разрежения,
    \item $p_2, u_2, \rho_2$ - условия в газе между волной разрежения и контактным разрывом (прошла волна разрежения)
    \item $p_3, u_3, \rho_3$ - условия в газе между контактным разрывом и ударной волной (прошла ударная волна)
    \item $p_R, u_R, \rho_R$ - начальные условия газа справа (газ, по которому еще не прошла ударная волна).
\end{itemize}

Решение состоит из соотношений на трех волнах: ударной волне, контактном разрыве и волне разрежения. 
\begin{enumerate}
    \item Для ударной волны. \\
    Соотношения Ранкина-Гюгонио выраженные через число Маха ($M= D/c_1$) выглядят следующим образом:
    \begin{eqnarray}
        \frac{p_3}{p_R} & = & \frac{2\gamma}{\gamma+1}M^2-\frac{\gamma-1}{\gamma+1} \\
        \frac{\rho_3}{\rho_R} & = & \frac{(\gamma+1)M^2}{(\gamma-1)M^2+2} \\
        \frac{u_3}{c_R} & = & \frac{2}{\gamma+1}(M-\frac{1}{M})
    \end{eqnarray} 
    $c_R = \sqrt{\gamma p_R / \rho_R}$  - скорость звука в среде перед фронтом, $D$ - скорость ударной волны.
    
    Формула адиабаты Ранкина-Гюгонио:
    \begin{align}
        \frac{p_3}{p_R}=\frac{\rho_3(\gamma+1)-\rho_R(\gamma-1)}{\rho_R(\gamma+1)-\rho_3(\gamma-1)}
    \end{align}

    \item Для контактного разрыва.
    Давление и скорость газа не меняются на контактном разрыве. 
    \begin{eqnarray}
        p_2&=&p_3\\
        \rho_2&=&\rho_L \left( \frac{p_3}{p_L} \right)^\frac{1}{\gamma}\\
        u_2&= &u_3
    \end{eqnarray} 
    \item Для волны разрежения.
    В волне разрежения параметры убывают непрерывно, в отличие от ударных волн.
    \begin{eqnarray}
        p_1(x,t)&=&p_L \left( 1-\frac{\gamma-1}{2}\frac{u_1(x,t)}{c_L} \right)^\frac{2\gamma}{\gamma-1},\\
        \rho_1(x,t)&=&\rho_L \left( 1-\frac{\gamma-1}{2}\frac{u_1(x,t)}{c_L} \right)^\frac{2}{\gamma-1},\\
        u_1(x,t) &=& \frac{2}{\gamma+1}(c_L + \frac{x}{t}).
    \end{eqnarray} 
\end{enumerate}
    
На рис.2 представлены графики параметров газа($p,\rho,u, S, T$) в начальный момент времени ($t=0$) и момент $t=0.2$ при начальных условиях задачи Сода: 
\begin{equation}
        (p, u, \rho) =
        \begin{cases}
            (1.0, 0.0, 1.0), & \text{при } -1\leq x\leq0,\\
            (0.1, 0.0, 0.125), & \text{при } 0\leq x\leq1.
        \end{cases}
\end{equation}
\clearpage
\begin{figure}[!htb]
	\centering
	\includegraphics[width=1\textwidth]{2 time moments (no time).png}
	\caption{
		Параметры газа в начальный момент $t=0$ (слева) и в момент $t=0.2$ (справа)
	}
	\label{fig:time_moments}
\end{figure}
\subsection{Задача Седова}
В задаче Седова или задаче о сильном точечном взрыве рассматривается распространение ударной волны большой мощности, возникшей в результате сильного взрыва (мгновенное выделение большого количества энергии в некотором объеме).
%---------------------------------------
\section{Механизмы радиоизлучения} \label{sec: Radio emission}
В данной работе основным рассматриваемым диапазоном электромагнитных волн является радиодиапазон.
К нему относятся ЭМ волны с длиной волны ($\lambda$) более 1мм (или частотами до 220 ГГц).
Радиоизлучение в космическом пространстве может появляеться по разным причинам.
К механизмам его возникновения относятся: тепловое радиоизлучение, реликтовое излучение, магнитотормозное (циклотронное и синхротронное) радиоизлучение, радиоизлучение плазмы, рекомбинационное (линейчатое) радиоизлучение, мазерное излучение молекул. 
В остатках сверхновоых осуществляются лишь некоторые из них. \cite{Kaplan1966}
\subsection{Синхротронное излучение} \label{sec: Synchrotron}
\indentОсновным механизмом возникновения радиоизлучения в остатках сверхновых является синхротронное излучение.
Оно имеет магнитотормозную природу, но отличается от циклотронного тем, что частицы здесь являются релитивистскими, соотвественно их энергии много больше. 
Для описания движения таких частиц используются законы теории относительности. Далее приводятся некоторые основные формулы и характеристики, описывающие данный тип излучения. 
Для частиц движущихся со скоростями v близкими к скорости света с энергия задается формулой: $$E = m_0c^2\cdot \gamma,$$ где $\gamma = {\frac{1}{\sqrt{1-v^2/c^2}}}$ - фактор Лоренца, $m_0$-масса неподвижной частицы. Также при движении тела его размер в направлении движения сокрщается в $\gamma$ раз и во столько же раз замедляется ход времени в нем. 
Как и в случае циклотронного излучения электрон в магнитном поле движется по окружности или по спирали. Но теперь его труднее "закручивать" - ведь масса электрона увеличилась в (E/0.51) раз, следовательно во столько же раз увеличится и радиус окружности, описываемой электроном, и во столько же раз будет меньше частота его обращения. Для релятивистского электрона: 
$$ \Tilde{f}_H=18\frac{H_{\perp}}{E}\hbox{ \slshape[кГц]},$$
здесь $H_{\perp} = Hcos\theta$ - компонента магнитного поля, перпендикулярная скорости электрона. 
Таким образом, основная частота обращения релятивистского электрона мала; поэтому велика длина волны и "основного тона" его радиоизлучения. Но релятивистский электрон значительно больше энергии излучает на высоких обертонах(гармониках). Дело здесь в следующем. Неподвижный электрон создает вокруг себя электрическое поле, одинаковое по всем направлениям, а если он движется с ускорением, но медленно, то вместе с ним движется и это сферически симметричное поле. Поэтому медленный электрон излучает более или менее одинаково во всех направлениях. Если же электрон движется очень быстро, то его электрическое поле как бы сплющивается в направлении движения из-за сокращения масштабов. Это означет, что поле особенно сильно меняется в направлении вдоль скорости электрона; отсюда также следует, что релятивистский электрон излучает электромагнитные волны главным образом вперед, по направлению своего движения. 

\begin{figure}[!htb]
	\centering
	\includegraphics[width=0.5\textwidth]{synchrotron_radiation.png}
	\caption{
		К объяснению синхротронного механизма радиоизлучения.
	}
	\label{fig:synchrotron_radiation}
\end{figure}
Угол раствора конуса, в который излучает релятивистский электрон, по порядку величины (в радианах) равен тому же универсальному отношению 0.51/E (МэВ).
%------------------------------------------------

\clearpage
\sloppy
\printbibliography

%-------------------------------------------------


\end{document}
